\documentclass[11pt,a4paper]{article}

\usepackage[utf8]{inputenc} % Zeichenkodierung
\usepackage[T1]{fontenc}    % Bessere Zeichenunterstützung
\usepackage[german]{babel}  % Korrekte deutsche Typografie
\usepackage[a4paper, left=2cm, right=2cm, top=2.5cm, bottom=2.5cm]{geometry}

\usepackage{graphicx}
\usepackage[export]{adjustbox}
\usepackage{microtype}
\usepackage{epstopdf}
\usepackage{float}
\usepackage{amsmath}
\usepackage{amssymb}
\usepackage{amstext}
\usepackage{amsfonts}
\usepackage{mathrsfs}
\usepackage{hyphenat}
\usepackage{fancyhdr}  
\usepackage{qrcode}  % QR-Code generieren

\title{}
\date{}  % Kein Datum auf Standard-Titel

\begin{document}

% === DECKBLATT ===
\begin{titlepage}
    \centering
    {\Huge Netzwerke und Schaltungen II}\\[0.8cm]
    {\Large D-ITET}\\[0.8cm]
    {\Large HS2025}\\[3.5cm]
    
    {\Huge \textbf{Übung X}}\\[1cm]
    {\Large 1.1.2000}\\[3.5cm]

    \qrcode[height=6cm]{https://www.example.com}

    \vfill
    {\Large \textbf{Rares Sahleanu}}
\end{titlepage}

% === INHALTSVERZEICHNIS ===
\newpage
\tableofcontents

% === AKTIVIERT KOPF- UND FUSSZEILEN AB SEITE 3 ===
\newpage
\pagestyle{fancy}  

% Kopfzeile: Links - Mitte - Rechts (subtil & kleiner)
\fancyhead[L]{\small \textit{Netzwerke und Schaltungen II}}  
\fancyhead[C]{\small \textit{Übung X}}
\fancyhead[R]{\small \textit{Rares Sahleanu}}

% Fußzeile: Seitenzahl in der Mitte
\fancyfoot[L]{}  
\fancyfoot[C]{\thepage}
\fancyfoot[R]{}

% Linie in der Kopfzeile aktivieren, Fußzeile ohne Linie
\renewcommand{\headrulewidth}{0.4pt}  
\renewcommand{\footrulewidth}{0pt}  

% === START DES DOKUMENTS ===
\section{Grundlagen der Netzwerkanalyse}
Die Netzwerkanalyse beschäftigt sich mit der Berechnung von Strömen und Spannungen in elektrischen Netzwerken. Wichtige Konzepte sind:

\subsection{Kirchhoffsche Gesetze}
Die Kirchhoffsche Regeln werden zur Analyse von elektrischen Netzwerken verwendet.

\subsection{Maschen- und Knotenanalyse}
Die Maschen- und Knotenanalyse ist eine wichtige Methode zur Netzwerkanalyse.

\subsection{Zweipoltheorie}
Ein elektrisches Zweipolnetz kann als Thevenin- oder Norton-Ersatzschaltung modelliert werden.

\subsection{Zusatzaufgaben}
\begin{itemize}
    \item Aufgabe 1: Berechnen Sie die Spannungen in einem einfachen Widerstandsnetzwerk mit zwei Maschen.
    \item Aufgabe 2: Verwenden Sie die Knotenpunktanalyse, um die Ströme in einem Netzwerk mit drei Widerständen und einer Spannungsquelle zu bestimmen.
\end{itemize}
\vspace{1cm}
% === ZWEITES KAPITEL ===
\section{Frequenzgang und Filter}
Der Frequenzgang eines Netzwerks beschreibt die Abhängigkeit der Übertragungsfunktion von der Frequenz.

\subsection{Tiefpass- und Hochpassfilter}
Tiefpass- und Hochpassfilter ermöglichen die Frequenzselektion.

\subsection{Bandpass- und Bandsperrfilter}
Bandpass- und Bandsperrfilter entfernen spezifische Frequenzbereiche.

\subsection{Bode-Diagramme}
Bode-Diagramme stellen den Frequenzgang von Systemen grafisch dar.

\subsection{Zusatzaufgaben}
\begin{itemize}
    \item Aufgabe 1: Bestimmen Sie die Grenzfrequenz eines einfachen RC-Tiefpassfilters.
    \item Aufgabe 2: Zeichnen Sie das Bode-Diagramm eines gegebenen RLC-Bandpassfilters.
\end{itemize}

\end{document}
