\documentclass{article}
\usepackage[margin=1in]{geometry} % Moderat eingestellte Ränder
\usepackage{circuitikz}
\usepackage{amsmath}
\usepackage{siunitx} % Für Einheiten, optional
\ctikzset{voltage/direction=straight}

\begin{document}

\section*{Aufgabe: Bestimmung des Stroms und Zeichnen des Phasors}

Gegeben ist ein Wechselstromkreis mit einer sinusförmigen Spannungsquelle 
\[
U(t)=10\,\sin(100t)\,\text{V}
\]
und einer Impedanz
\[
Z=4+j3\,\Omega.
\]
Bestimmen Sie den komplexen Strom \(I\) im Stromkreis und zeichnen Sie den zugehörigen Phasor.

\bigskip

\begin{center}
\begin{circuitikz}[european]
    % Wechselspannungsquelle mit Beschriftung
    \draw (0,0) to[sV, v={$U(t)=10\sin(100t)$}] (0,4);
    % Verbindung zur Impedanz
    \draw (0,4) -- (3,4)
          to[R, l={$Z=4+j3\,\Omega$}] (3,0) -- (0,0);
\end{circuitikz}
\end{center}



\section*{Aufgabe: Bestimmung des Stroms und Zeichnen der Spannungszeiger}

Gegeben ist ein Wechselstromkreis mit einer sinusförmigen Stromquelle
\[
I(t)=2\,\sin(100t)\,\text{A}
\]
und zwei seriellen Impedanzen
\[
Z_1=2+j1\,\Omega,\quad Z_2=3+j2\,\Omega.
\]
Bestimmen Sie den komplexen Strom \(I\) im Stromkreis und zeichnen Sie die abfallenden Spannungszeiger (Phasoren) an \(Z_1\) und \(Z_2\).

\bigskip

\begin{center}
\begin{circuitikz}[european, straight voltages]
    % Stromquelle
    \draw (0,0) to[I, i={$I(t)=2\sin(100t)$}] (0,4);
    % Verbindung zur ersten Impedanz
    \draw (0,4) -- (3,4);
    % Erste Impedanz mit Spannungspfeil
    \draw (3,4) to[R, l={$Z_1=2+j1\,\Omega$}, v={$U_{Z_1}$}] (6,4);
    % Zweite Impedanz mit Spannungspfeil
    \draw (6,4) to[R, l={$Z_2=3+j2\,\Omega$}, v={$U_{Z_2}$}] (9,4);
    % Rückführung zum Ausgangspunkt
    \draw (9,4) -- (9,0) -- (0,0);
\end{circuitikz}
\end{center}
\newpage
\section*{Aufgabe: Berechnung von \(i_2\)}

Gegeben ist ein Stromteiler mit einer sinusförmigen Stromquelle und zwei parallelen Zweigen.  
Im ersten Zweig befinden sich ein Widerstand \(R_1\) und eine Spule \(L_1\) in Serie, im zweiten Zweig eine Spule \(L_2\), ein Widerstand \(R_2\), ein Kondensator \(C\), ein Widerstand \(R_3\) und ein weiterer Widerstand \(R_4\) in Serie.  
Die Ströme lauten:
\[
i \quad \text{(Hauptstrom)},\quad i_1 \quad \text{(Zweig 1)},\quad i_2 \quad \text{(Zweig 2)}.
\]
Bestimmen Sie \(i_2\).

\bigskip

\begin{center}
\begin{figure}[!ht]
\centering
% Linkes Bild: Erstes Schaltbild
\begin{minipage}{0.49\textwidth}
\centering
\resizebox{\textwidth}{!}{%
\begin{circuitikz}
\tikzstyle{every node}=[font=\large]
\draw (17.75,13.75) to[C,l={\large $C_1$}] (17.75,10.5);
\draw (17.75,10.5) to[L,l={\large $L_1$}] (17.75,8.75);
\draw (17.75,8.75) to[C,l={\large $C_1$}] (17.75,7.25);
\draw (15,12.75) to[L,l={\large $L_2$}] (15,10.5);
\draw (17.75,14.25) to[european resistor,l={\large $R_1$}] (17.75,16);
\draw (15,12.75) to[european resistor,l={\large $R_2$}] (15,16);
\draw (15,9.75) to[european resistor] (15,6.5);
\draw (17.75,7.25) to[short] (17.75,6.5);
\draw (17.75,6.5) to[short] (10.25,6.5);
\draw (17.75,16) to[short] (10.25,16);
\draw (10.25,16) to[sinusoidal voltage source, sources/symbol/rotate=auto] (10.25,6.5);
\draw (15,9.75) to[short] (15,10.5);
\draw (17.75,13.25) to[short] (17.75,14.25);
% Pfeile für Stromrichtungen
\draw [->, >=Stealth] (15,13.25) -- (15,13) node[midway, right, fill=white] {$i_2$};
\draw [->, >=Stealth] (13.5,16) -- (13.75,16) node[midway, above, fill=white] {$i_{\text{haupt}}$};
\draw [->, >=Stealth] (16,16) -- (16.5,16) node[midway, above, fill=white] {$i_1$};
\draw [->, >=Stealth] (9.25,12.25) -- (9.25,10.5) node[midway, left, fill=white] {u(t)};
\end{circuitikz}
}
\end{minipage}\hfill
% Rechtes Bild: Zweites Schaltbild
\begin{minipage}{0.49\textwidth}
\centering
\resizebox{\textwidth}{!}{%
\begin{circuitikz}
\tikzstyle{every node}=[font=\large]
\draw (17,11.75) to[european resistor,l={\large $Z_1$}] (17,8.75);
\draw (15,11.75) to[european resistor,l={\large $Z_2$}] (15,8.75);
\draw (17,11.75) to[short] (13.25,11.75);
\draw (17,8.75) to[short] (13.25,8.75);
\draw (13.25,11.75) to[sinusoidal voltage source, sources/symbol/rotate=auto] (13.25,8.75);
% Pfeile für Stromrichtungen
\draw [->, >=Stealth] (12.75,10.75) -- (12.75,9.5) node[midway, left, fill=white] {u(t)};
\draw [->, >=Stealth] (15,11.5) -- (15,11) node[midway, right, fill=white] {$i_2$};
\draw [->, >=Stealth] (15.5,11.75) -- (16,11.75) node[midway, above, fill=white] {$i_1$};
\draw [->, >=Stealth] (14,11.75) -- (14.25,11.75) node[midway, above, fill=white] {$i_{\text{haupt}}$};
\end{circuitikz}
}
\end{minipage}
\caption{Linkes Bild: Stromteiler-Schaltung; Rechtes Bild: Impedanzschaltung}
\label{fig:combined}
\end{figure}
\end{center}

\section*{Aufgabe: Spannungsteiler mit Widerstand und Kondensator}

Gegeben ist ein Spannungsteiler, der aus einem Widerstand \(R\) und einem Kondensator \(C\) besteht.  
An den Spannungsteiler wird die sinusförmige Spannung
\[
u(t)=U_0\sin(\omega t)
\]
angelegt. Bestimmen Sie den Spannungsabfall \(u_C(t)\) am Kondensator in Abhängigkeit von \(R\), \(C\) und \(\omega\).

\bigskip

\begin{center}
\begin{circuitikz}[american]
  % Zeichne die sinusförmige Spannungsquelle, den Widerstand und den Kondensator in Serie
  \draw (0,0) to[sinusoidal voltage source, l={$u(t)=U_0\sin(\omega t)$}] (0,4)
        to[R, l={$R$}] (3,4)
        to[C, l={$C$}, v^={$u_C(t)$}] (3,0) -- (0,0);
\end{circuitikz}
\end{center}
\end{document}
